\section*{Exercise 6.1 : An electromagnetic relay}
\underline{1. State transformation :}
We have the initial state variables :
\begin{align*}
x_1 & = z \\
x_2 & = \dot{z} \\
x_3 & = I 
\end{align*}
We introduce the following new state variables :
\begin{align*}
y_1 & = z \\
y_2 & = \dot{z} \\
y_3 & = \phi(z,I)
\end{align*}
where $$\phi(z,I) = \frac{\alpha I}{1+\beta z}.$$
To go from the initial state variables $x$ to the new state variables $y$, we have the following non-linear state transformation $T$ : $$y = T(x)$$ with
\begin{align*}
y_1 & = x_1 \\
y_2 & = x_2 \\
y_3 & = \phi(x_1,x_3) = \frac{\alpha x_3}{1+\beta x_1}
\end{align*}
We would like to show that this transformation is valid.\\
For this, two conditions must be satisfied.
\begin{enumerate}
\item The application $T$ has to be bijective, that is that it must exist an inverse function $T^{-1}$ so that $x = T^{-1}(y)$.
\item $T$ and $T^{-1}$ have to be continuous and differentiable functions.
\end{enumerate}
Let's search for the inverse application $T^{-1}$.
$$x = T^{-1}(y) = T^{-1}(T(x)) = T^{-1}
\begin{pmatrix}
x_1 \\
x_2 \\
\frac{\alpha x_3}{1+\beta x_1}
\end{pmatrix}.$$
We then find that the inverse application $T^{-1}$ is the following :
$$T^{-1}
\begin{pmatrix}
y_1 \\
y_2 \\
y_3
\end{pmatrix}
=
\begin{pmatrix}
y_1 \\
y_2 \\
\frac{1+\beta y_1}{\alpha}y_3
\end{pmatrix}.$$
We are now going to check the second condition. \\
Let's calculate the Jacobian matrices of $T$ and $T^{-1}$. We then have
$$\frac{\partial T}{\partial x} =
\begin{pmatrix}
1 & 0 & 0 \\
0 & 1 & 0 \\
\frac{\alpha \beta x_3}{(1+\beta x_1)^2} & 0 & \frac{\alpha}{1+\beta x_1}
\end{pmatrix}, \quad
\frac{\partial T^{-1}}{\partial y} =
\begin{pmatrix}
1 & 0 & 0\\
0 & 1 & 0\\
\frac{\beta}{\alpha}y_3 & 0 & \frac{1+\beta y_1}{\alpha}
\end{pmatrix}.$$
We get that the applications $T$ et $T^{-1}$ are continuous and differentiable when $(1+\beta x_1) \neq 0$ and $\alpha \neq 0$. Since $x_1 = z$ characterize a distance, that variable is always positive. By taking $\alpha > 0$ and $\beta > 0$, we are sure that the transformation is valid.\\
Let's now establish the state representation of these new variables. The state representation of the initial variables was the following :
\begin{align*}
\dot{x}_1 & = x_2, \\
\dot{x}_2 & = \frac{k}{m}(z_0-x_1)-\frac{\alpha \beta}{2m}\Big(\frac{x_3}{1+\beta x_1}\Big)^2, \\
\dot{x}_3 & = \frac{\beta x_2 x_3}{1+\beta x_1}-\frac{R}{\alpha}(1+\beta x_1)x_3 + \frac{1+\beta x_1}{\alpha}u.
\end{align*}
We are then going to apply the application $T$ to the variables $x_1,x_2$ and $x_3$. As a reminder, we previously had :
$$\begin{pmatrix}
y_1\\
y_2\\
y_3\\
\end{pmatrix}
=T
\begin{pmatrix}
x_1\\
x_2\\
x_3\\
\end{pmatrix}
=
\begin{pmatrix}
x_1 \\
x_2 \\
\frac{\alpha x_3}{1+\beta x_1}
\end{pmatrix}.$$
The state representation of the new variables is then
\begin{align*}
\dot{y}_1 & = y_2, \\
\dot{y}_2 & = \frac{k}{m}(z_0-y_1)-\frac{\alpha \beta}{2m}y_3^2, \\
\dot{y}_3\frac{1+\beta y_1}{\alpha} + y_3\frac{\beta \dot{y}_1}{\alpha} & = \frac{\beta y_2 y_3}{\alpha}-\frac{R}{\alpha}(1+\beta y_1)\frac{(1+\beta y_1)y_3}{\alpha} + \frac{1+\beta y_1}{\alpha}u.
\end{align*}
By isolating $\dot{y}_3$ in the last equation, we finally have the following state representation :
\begin{align*}
\dot{y}_1 & = y_2, \\
\dot{y}_2 & = \frac{k}{m}(z_0-y_1)-\frac{\alpha \beta}{2m}y_3^2, \\
\dot{y}_3 & = -\frac{R}{\alpha}(1+\beta y_1)y_3 + u.
\end{align*}

\underline{2. Brunovski canonical form :}
We are going to show that this system can be put under a Brunovski canonical form. By drawing the graph of the mono-input system, we realize that the shortest path from $u$ to $y_1$ has a length $3$, which is the size of the system. Using Definition 6.10, we conclude that this system is triangular. We can then write it under a Brunovski canonical form.\\
We are now going to determine the state transformation leading to a Brunovski canonical form. As a reminder, a mono-input system is under a Brunovski canonical form it has the follow structure :
\begin{align*}
\dot{z}_1 & = z_2, \\
\dot{z}_2 & = z_3, \\
\vdots \\
\dot{z}_n & = \alpha(z_1,\hdots,z_n,u).
\end{align*}
We just have to define
\begin{align*}
z_1 & = y_1,\\ 
z_2 & = y_2,\\
z_3 & = \frac{k}{m}(z_0-y_1)-\frac{\alpha \beta}{2m}y_3^2,\\
\end{align*}
for the model to be rewritten
\begin{align*}
\dot{z}_1 & = z_2,\\
\dot{z}_2 & = z_3,\\
\dot{z}_3 & = \hdots,
\end{align*}
which is a Brunovski canonical form.

\section*{Exercise 6.2 : A biochemical reactor}
\underline{1. State representation :}
The general form of the evolution of every entities inside the reactor is given by :
\begin{align*}
\dot{x}_A & = (\delta_A-\gamma_A)r(x_A,x_B)+\frac{1}{V}(Q_{0A}-Q_{A0})\\
\dot{x}_B & = (\delta_B-\gamma_B)r(x_A,x_B)+\frac{1}{V}(Q_{0B}-Q_{B0})
\end{align*}
where\\
$x_i$ is the concentration of entity $i$ inside the reactor.\\
$\delta_i$ is the coefficient of the entity $i$ in the products of the reaction.\\
$\gamma_i$ is the coefficient of the entity $i$ in the reactants of the reaction.\\
$r(x)$ is the speed of the reaction.\\
$V$ is the volume of the reactor.\\
$Q_{0i}$ is the flux of entity $i$ leaving the reactor.\\
$Q_{i0}$ is the flux of entity $i$ entering the reactor.\\
In our case, we work with a constant volume and we assume that the kinetics  obey the law of mass action, that is $$r(x_A,x_B) = k (x_A)^{\gamma_A} (x_B)^{\gamma_A}$$ where $k$ is the speed constant of the reaction.\\
Let $F$ be the entering volumetric rate of flow et let $c$ be its concentration in $A$.\\
From the chemical equation $$A+B \longrightarrow 2B,$$ we deduce
\begin{align*}
\delta_A & = 0\\
\delta_B & = 2\\
\gamma_A & = 1\\
\gamma_B & = 1\\
\end{align*}
We then have the following state representation :
\begin{align*}
\dot{x}_A & = (0-1)k x_A x_B+\frac{1}{V}(Fc-Fx_A)\\
\dot{x}_B & = (2-1)k x_A x_B+\frac{1}{V}(0-Fx_B)
\end{align*}
By setting $u = F/V$, the volumetric inflowing rate, the state representation has finally the following form :
\begin{align*}
\dot{x}_A & = -k x_A x_B+u(c-x_A)\\
\dot{x}_B & = k x_A x_B-ux_B
\end{align*}

\underline{Conservative system :}
Let $C$ be the vector with the coefficients in front of $r(x)$ in the state representation. In our case, we have 
$$C = \begin{pmatrix}
-1\\
1
\end{pmatrix}.$$
In order to show that the system is conservative, we have to find a vector $\omega$ in which all the entries are strictly positives and which is in the kernel of $C^{\top}$. In the present case, we can, for example, choose
$$\omega = \begin{pmatrix}
1\\
1
\end{pmatrix}$$ because we have $$C^{\top}\omega = -1+1 = 0.$$
This implies that, when $u = 0$, we have $$\frac{d(x_A+x_B)}{dt} = \omega_A \dot{x}_A + \omega_B \dot{x}_B = [\omega^{\top}C]r(x) = 0.$$

\underline{3. Brunovski canonical form :}
Let's search for tow new state variables $z_1$ and $z_2$ such that the system takes the following form :
\begin{align*}
\dot{z}_1 & = z_2\\
\dot{z}_2 & = \alpha(z_1,z_2,u)
\end{align*}
where $\alpha$ is some function.\\
Setting $z_1 = f(x_A,x_B)$, we then have 
\begin{align*}
\dot{z}_1 & = \frac{\partial f}{\partial x_A}\dot{x}_A + \frac{\partial f}{\partial x_B}\dot{x}_B\\
 & = \frac{\partial f}{\partial x_A}(-kx_Ax_B) + \frac{\partial f}{\partial x_B}(kx_Ax_B) + u\Big(\frac{\partial f}{\partial x_A}(c-x_A)-\frac{\partial f}{\partial x_B}x_B\Big).
\end{align*}
Since we don't want $\dot{z}_1$ to depend on the input $u$, we will force $$\frac{\partial f}{\partial x_A}(c-x_A)-\frac{\partial f}{\partial x_B}x_B = 0.$$
An example of function $f$ for which this is true is $$f(x_A,x_B) = \frac{x_B}{c-x_A}.$$
By setting $$z_2 = \frac{\partial f}{\partial x_A}(-kx_Ax_B) + \frac{\partial f}{\partial x_B}(kx_Ax_B) = \frac{kx_Ax_B(c-x_A-x_B)}{(c-x_A)^2},$$ we then get the following Brunovski canonical form :
\begin{align*}
\dot{z}_1 & = z_2\\
\dot{z}_2 & = ...
\end{align*}

\underline{4. Reversible reaction :}
The chemical reaction is $$A + B \longleftrightarrow 2B$$ which equivalent to the two following reactions acting simultaneously
\begin{align*}
A + B & \longrightarrow 2B\\
2B & \longrightarrow A+B
\end{align*}
Thus, the state representation of the reversible system is
\begin{align*}
\dot{x}_A & = -k_1 x_A x_B + k_2 x_B^2 + u(c-x_A)\\
\dot{x}_B & = k x_A x_B - k_2 x_B^2 - ux_B
\end{align*}
where $k_1$ is the speed constant of the first reaction and $k_2$ is the speed constant of the second one.\\
We've already seen that when there's only the direct reaction, the system is conservative. If there's only the inverse direction, we have 
$$C = 
\begin{pmatrix}
1\\
-1
\end{pmatrix}$$ and by choosing
$$\omega =
\begin{pmatrix}
1\\
1
\end{pmatrix},$$ we show that the corresponding system is conservative.\\
Since the system is conservative for the reaction in both directions, it is conservative for the reversible reaction.\\
By choosing the same transformation as for the irreversible case, i.e. $$f(x_A,x_B) = \frac{x_B}{c-x_A},$$ we find the following new variables
\begin{align*}
z_1 & = \frac{x_B}{c-x_A}\\
z_2 & = \frac{(k_1x_Ax_B-k_2x_B^2)(c-x_A-x_B)}{(c-x_A)^2}.
\end{align*} 
We then get the following Brunovski canonical form :
\begin{align*}
\dot{z}_1 & = z_2\\
\dot{z}_2 & = ...
\end{align*}

\section*{Exercise 6.3 : A two compartments system}
The state representation of this two compartments linear system is given by the following general form :
\begin{align*}
\dot{x}_1 & = k_{21}x_2-k_{10}x_1-k_{12}x_1+k_{01}u\\
\dot{x}_2 & = k_{12}x_1-k_{20}x_2-k_{21}x_2
\end{align*}
where the $k_{ij} > 0$ are the constants characterizing the incoming and outgoing flux. \\
Using matrices, the system can be rewritten
$$\begin{pmatrix}
\dot{x}_1\\
\dot{x}_2
\end{pmatrix}=
\begin{pmatrix}
-k_{10}-k_{12} & k_{21} \\
k_{12} & -k_{20}-k_{21}
\end{pmatrix}
\begin{pmatrix}
x_1\\
x_2
\end{pmatrix}+
\begin{pmatrix}
k_{01}\\
0
\end{pmatrix}u$$
or, in a simplified notation,
$$\dot{x} = Ax + Bu.$$
Let's begin by determining the eigenvalues of $A$. They're the solutions of the equation
\begin{align*}
\det(A-\lambda I)=0 & \iff (k_{10}+k_{12}+\lambda)(k_{20}+k_{21}+\lambda)-k_{12}k_{21}=0\\
 & \iff \lambda^2 + (k_{10}+k_{12}+k_{20}+k_{21})\lambda + (k_{10}k_{20}+k_{10}k_{21}+k_{12}k_{20}) = 0
\end{align*}
We then find
\begin{align*}
\lambda_1 & = \frac{-(k_{10}+k_{12}+k_{20}+k_{21}) + \sqrt{(k_{10}+k_{12}-k_{20}-k_{21})^2+4k_{12}k_{21}}}{2}\\
\lambda_2 & = \frac{-(k_{10}+k_{12}+k_{20}+k_{21}) - \sqrt{(k_{10}+k_{12}-k_{20}-k_{21})^2+4k_{12}k_{21}}}{2}.
\end{align*}
Since $k_{ij}>0$, the member in the square root is always strictly positive. This implies that the two roots of the characteristic polynomial, which are the two eigenvalues of $A$, are always distinct.\\
Let's call $T$ the linear state transformation such as $z = Tx$ where $z$ is the vector of the new state variables. Under the transformation $T$, the above system would be written $$T\dot{x} = TAx + TBu \iff \dot{z} = TAT^{-1}z + TBu.$$
We are searching for a linear transformation $T$ which diagonalize the initial system. In other words, we are searchin for $T$ such as $TAT^{-1}$ is a diagonal matrix. To get there, we need to determine the eigenvectors of $A$. That is two vectors $v$ linearly independent, verifying
$$\begin{pmatrix}
-k_{10}-k_{12} & k_{21} \\
k_{12} & -k_{20}-k_{21}
\end{pmatrix}
\begin{pmatrix}
v_a\\
v_b
\end{pmatrix}=
\lambda
\begin{pmatrix}
v_a\\
v_b
\end{pmatrix}
\iff
\begin{pmatrix}
(-k_{10}-k_{12})v_a + k_{21}v_b\\
k_{12}v_a + (-k_{20}-k_{21})v_b
\end{pmatrix}=
\lambda
\begin{pmatrix}
v_a\\
v_b
\end{pmatrix}.$$
By setting $v_a = k_{21}$ is the first equation and $v_b=k_{12}$ in the second one, the two following eigenvectors of $A$ : 
$$v_1=\begin{pmatrix}
k_{21}\\
\lambda_1 +k_{10} + k_{12}
\end{pmatrix}, \quad
v_2 =
\begin{pmatrix}
\lambda_2 +k_{20} + k_{21}\\
k_{12}
\end{pmatrix}.$$
We are assured that these two vectors are linearly independent since the determinant of the eigenvectors matrix equals zero if and only if $\lambda_1=\lambda_2$ which is never the case.\\
We then have
$$A\begin{pmatrix}
v_1 & v_2 
\end{pmatrix}=
\begin{pmatrix}
\lambda_1 v_1 & \lambda_2 v_2
\end{pmatrix}=
\begin{pmatrix}
v_1 & v_2 
\end{pmatrix}
\begin{pmatrix}
\lambda_1 & 0\\
0 & \lambda_2
\end{pmatrix}.$$
In other to link with what was done previously, we have 
$$T = \begin{pmatrix}
v_1 & v_2 
\end{pmatrix}^{-1}.$$
By setting $K = (\lambda_1 +k_{10} + k_{12})(\lambda_2 +k_{20} + k_{21})-k_{12}k_{21}$, we find
$$T = \frac{1}{K}
\begin{pmatrix}
-k_{12} & (\lambda_2 +k_{20} + k_{21})\\
(\lambda_1 +k_{10} + k_{12}) & -k_{21}
\end{pmatrix}.$$
The above transformation $T$ is the one that diagonalize the system.\\
Let's now find the time constants which are given by $\tau_i = |\lambda_i|^{-1},i=1,\hdots,n$. In our case, the eigenvalues of the system were
\begin{align*}
\lambda_1 & = \frac{-(k_{10}+k_{12}+k_{20}+k_{21}) + \sqrt{(k_{10}+k_{12}-k_{20}-k_{21})^2+4k_{12}k_{21}}}{2}\\
\lambda_2 & = \frac{-(k_{10}+k_{12}+k_{20}+k_{21}) - \sqrt{(k_{10}+k_{12}-k_{20}-k_{21})^2+4k_{12}k_{21}}}{2}.
\end{align*}
We verify that these values are both strictly negative. We then have
\begin{align*}
|\lambda_1| & = \frac{(k_{10}+k_{12}+k_{20}+k_{21}) - \sqrt{(k_{10}+k_{12}-k_{20}-k_{21})^2+4k_{12}k_{21}}}{2}\\
|\lambda_2| & = \frac{(k_{10}+k_{12}+k_{20}+k_{21}) + \sqrt{(k_{10}+k_{12}-k_{20}-k_{21})^2+4k_{12}k_{21}}}{2}.
\end{align*}
Therefore, we find the following time constants :
\begin{align*}
\tau_1 = |\lambda_1|^{-1} & = \frac{2}{(k_{10}+k_{12}+k_{20}+k_{21}) - \sqrt{(k_{10}+k_{12}-k_{20}-k_{21})^2+4k_{12}k_{21}}}\\
\tau_2 = |\lambda_2|^{-1} & = \frac{2}{(k_{10}+k_{12}+k_{20}+k_{21}) + \sqrt{(k_{10}+k_{12}-k_{20}-k_{21})^2+4k_{12}k_{21}}}.
\end{align*}